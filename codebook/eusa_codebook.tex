%--------------------------------------------------%
% generated by the codebookr R package
% created by Joshua C. Fjelstul, Ph.D.
%--------------------------------------------------%

\documentclass[10pt]{article}

%--------------------------------------------------%
% packages
%--------------------------------------------------%

% page layout
\usepackage{geometry}

% fonts
\usepackage[english]{babel}
\usepackage{underscore}
\usepackage{anyfontsize}
\usepackage[utf8]{inputenc}
\usepackage[T1]{fontenc}
\usepackage{fontspec}

% graphics and tables
\usepackage{graphicx} % add figures
\usepackage{xcolor} % change font color
\usepackage{tikz} % add graphics

% paragraph spacing
\usepackage{setspace}

% hyperlinks
\usepackage{url}

% table of contents
\usepackage{tocloft}

% test alignment
\usepackage{ragged2e}

% multi-page tables
\usepackage{longtable}

% custom lists
\usepackage{enumitem}

% insert content on every page
\usepackage{atbegshi} 

% code formatting
\usepackage{tcolorbox}

%--------------------------------------------------%
% colors
%--------------------------------------------------%

% define colors
\definecolor{themecolor}{HTML}{4D9FEB}
\definecolor{background}{HTML}{EEF6FD}

% format hyperlinks
\usepackage[colorlinks=true,linkcolor=themecolor,citecolor=themecolor,urlcolor=themecolor,breaklinks=true]{hyperref}

%--------------------------------------------------%
% formatting
%--------------------------------------------------%

% configure main font
\setmainfont[Ligatures=TeX,BoldFont={Roboto Medium}]{Roboto Light}
\setmonofont[Ligatures=TeX]{Roboto Mono-Light}

% set page margins
\geometry{top = 1.5in, bottom = 1.5in, left = 1.5in, right = 1.5in}

% set paper size
\geometry{letterpaper}

% format table of contents
\renewcommand{\cftsecdotsep}{10}
\renewcommand{\cftsecleader}{\cftdotfill{\cftdotsep}}
\renewcommand{\cftsecfont}{{\small\color{black!75}\bfseries}}
\renewcommand{\cftsecpagefont}{{\small\color{black!75}\normalfont}}

% adjust spacing
\usepackage{parskip}
\parskip=10pt
\renewcommand{\baselinestretch}{1.4}

% hyphen formatting
\hyphenpenalty = 10000
\exhyphenpenalty = 10000

% prevent widow and orphan lines
\widowpenalty10000
\clubpenalty10000

%--------------------------------------------------%
% page elements
%--------------------------------------------------%

% a command to make a code box
\newtcbox{\codebox}{nobeforeafter,tcbox raise base,colback=black!5,colframe=white,coltext=black!75,boxrule=0pt,arc=3pt,boxsep=0pt,
left=4pt,right=4pt,top=3pt,bottom=3pt}

% a command to make a chip
\newtcbox{\chip}{nobeforeafter,tcbox raise base,colback=black!5,colframe=white,coltext=black!75,boxrule=0pt,arc=11pt,boxsep=0pt,
left=10pt,right=10pt,top=8pt,bottom=8pt}

% command to format code
\newcommand{\code}[1]{\codebox{{\footnotesize\texttt{#1}}}}

% command to highlight text
\newcommand{\highlight}[1]{{\color{themecolor} \textbf{#1}}}

% command to create a divider
\newcommand{\dividerline}{{\color{gray!10} \rule[4pt] {\textwidth}{3pt}}}

% command to add a cover
\newcommand{\cover}[4]{
\begin{tikzpicture}[remember picture,overlay, shift={(current page.south west)}]
\fill[themecolor] (0, 5.5in) rectangle ++ (8.5in, 5.5in); % header bar
\fill[black!5] (0, 4in) rectangle ++ (8.5in, 1.5in); % middle bar
\fill[white] (0, 0in) rectangle ++ (8.5in, 4in); % footer bar
\node[anchor=west] at (1.5in, 6.25in) {\color{white} \fontsize{60}{60}\selectfont \begin{minipage}{5.5in} \textbf{Codebook} \fontsize{15}{15}\selectfont \hspace{5pt} v #2 \end{minipage}};
\node[anchor=west, align=left] at (1.5in, 4.75in) {\begin{minipage}{5.5in} \color{black!40} \fontsize{#4}{#4} \selectfont #1 \end{minipage}};
\node[anchor=west, align=left, minimum height=2in] at (1.5in, 2.55in) {\begin{minipage}[t][2in]{5.5in} \color{black!40} \fontsize{10}{10} \selectfont #3 \end{minipage}};
\end{tikzpicture}
}

% command to add a header page
\newcommand{\headerpage}[4]{
	\newpage
	\begin{tikzpicture}[remember picture,overlay, shift={(current page.south west)}]
		\fill[themecolor] (0, 9in) rectangle ++ (8.5in, 2in); % header line 1
		\fill[black!5] (0, 8in) rectangle ++ (8.5in, 1in); % header line 2
		\node[anchor = west] at (1.5in, 9.6in) {\color{white} \fontsize{#3}{#3}\selectfont \textbf{#1}}; % heading
		\node[anchor = west] at (1.5in, 8.5in) {\color{black!40} \fontsize{#4}{#4}\selectfont #2}; % heading
	\end{tikzpicture}
	\phantomsection
	\addcontentsline{toc}{section}{#1}
	\vspace{1.5in}
}

% command to layout page
\newcommand\pagelayout{
	\begin{tikzpicture}[remember picture,overlay, shift={(current page.south west)}]
		% \fill[themecolor] (0, 10.75in) rectangle ++ (8.5in, 0.25in); % header
		\fill[black!5] (0, 0) rectangle ++ (8.5in, 0.5in); % footer
		\draw (0.25in, 0.25in) node[anchor = west] {\fontsize{9}{9}\selectfont \color{black!40} The EUSA Database Codebook \hspace{5pt} | \hspace{5pt} Joshua C. Fjelstul, Ph.D.}; % footer content
		\draw (8.25in, 0.25in) node[anchor = east] {\fontsize{9}{9}\selectfont \color{black!40} \thepage}; % page number
	\end{tikzpicture}
}

% add page layout 
\AtBeginShipout{
	\AtBeginShipoutUpperLeft{\pagelayout}
}

% command to add a subheading
\newcommand{\subheading}[1]{
\vspace{24pt}
{\color{themecolor} \fontsize{14}{14}\selectfont \textbf{#1}}
\vspace{6pt}
\dividerline
\vspace{-20pt}
}

%--------------------------------------------------%
% start document
%--------------------------------------------------%

\begin{document}

\clearpage
\pagestyle{empty}

\color{black!75}

\small

\begin{flushleft}

%--------------------------------------------------%
% cover
%--------------------------------------------------%

\cover{The European Union State Aid \\ (EUSA) Database}{1.0}{Joshua C. Fjelstul, Ph.D.}{16}

\newpage

%--------------------------------------------------%
% table of contents
%--------------------------------------------------%

% reset page counter
\setcounter{page}{1}

% format the table of contents header
% \renewcommand\contentsname{{\color{themecolor} \fontsize{14}{14}\selectfont Datasets}}
\renewcommand\contentsname{\subheading{Datasets} \vspace{0pt}}

% add the table of contents
\tableofcontents

% remove page number from table of contents pages
\addtocontents{toc}{\protect\thispagestyle{empty}}

\newpage

%--------------------------------------------------%
% content
%--------------------------------------------------%


%--------------------------------------------------%
% dataset
%--------------------------------------------------%

\headerpage{cases}{Case-level data}{30}{10}

\subheading{Description}

This dataset include data on state aid cases. There is one observation per case (1988-2020). The dataset includes information on the department responsible for the case, the member state that the case is against, and decisions taken by the Commission in the case.

\subheading{Variables}

\begin{description}[labelwidth=130pt, leftmargin=\dimexpr\labelwidth+\labelsep\relax, font=\normalfont, itemsep=10pt]
\item[\code{key\_id}] \code{numeric}\hspace{5pt}An ID number that uniquely identifies each observation in the dataset. 
\item[\code{case\_id}] \code{string}\hspace{5pt}An ID number that uniquely identifies each state aid case. Assigned by the Commission. The Commission changed the format of case numbers in 2010. Before this change, the case number indicated the type(s) of the procedure associated with the case. After this change, all case numbers have the format \code{SA.\#\#\#\#\#}.
\item[\code{procedure\_numbers}] \code{string}\hspace{5pt}Numbers that uniquely identifiy each procedure associated with a case. If there are multiple procedures, the procedure numbers are listed and separated by a comma. After the Commission changed the format of case numbers in 2010, only procedure numbers indicate the types of procedures associated with a case.
\item[\code{member\_state\_id}] \code{numeric}\hspace{5pt}An ID number that uniquely identifies each member state. This ID number is assigned when member states are sorted by accession date and then alphabetically. 
\item[\code{member\_state}] \code{string}\hspace{5pt}The name of the member state that the Commission opened the case against. 
\item[\code{member\_state\_code}] \code{string}\hspace{5pt}A two letter code assigned by the Commission that uniquely identifies each member state. 
\item[\code{department\_id}] \code{numeric}\hspace{5pt}An ID number that uniquely identifies each Directorate-General (DG) of the Commission. There are three DGs that can open state aid cases. The DG for Competition (COMP) is coded \code{1}, the DG for Agriculture and Rural Development (AGRI) is coded \code{2}, and the DG for Maritime Affairs and Fisheries (MARE) is coded \code{3}.
\item[\code{department}] \code{string}\hspace{5pt}The name of the Directorate-General (DG) of the Commission that opened the state aid case.
\item[\code{department\_code}] \code{string}\hspace{5pt}A multi-letter code assigned by the Commission that uniquely identifies each Directorate-General (DG) of the Commission. 
\item[\code{case\_type\_id}] \code{numeric}\hspace{5pt}An ID number that uniquely identifies each type of state aid cases. Coded \code{1} for cases that involve state aid measures granted through Commission-approved schemes that is not notifiable, coded \code{2} for cases that involve state aid measures granted through Commission-approved schemes where the Commission has required that the member state notify any aid granted through the scheme, and coded \code{3} for cases that involve ad hoc state aid measures that are not granted through a scheme.
\item[\code{case\_type}] \code{string}\hspace{5pt}The type of the state aid case. There are three types of cases. Coded \code{Scheme} for cases that involve state aid measures granted through Commission-approved schemes that is not notifiable, coded \code{Individual application} for cases that involve state aid measures granted through Commission-approved schemes where the Commission has required that the member state notify any aid granted through the scheme, and coded \code{Ad hoc} for cases that involve ad hoc state aid measures that are not granted through a scheme.
\item[\code{procedure\_types}] \code{string}\hspace{5pt}A list of the types of procedures associated with the case, separated by a comma. Procedure types are listed in the order that the Commission opened the procedures. Possible values include: \code{Contradictory aid}, for state aid that the Commission suspects violates the rules of the internal market, as defined by Article 107 of the Treaty of the Functioning of the European Union (TFEU); \code{Existing aid}, for existing state aid measures; \code{Notified aid}, for state aid that the member state has notified to the Commission; \code{Unnotified aid}, for state aid that the member state has not notified to the Commission; \code{General block exemption}, for state aid that is exempt from notification under the General Block Exemption Regulation (GBER); and \code{Special block exemption}, for state aid that is exempt from notification under a specific block exemption.
\item[\code{contradictory\_aid}] \code{dummy}\hspace{5pt}A dummy variable indicating whether the case relates to whether the Commission suspects that the state aid measure is contradictory to the rules of the single market, as defined by Article 107 of the Treaty of the Functioning of the European Union (TFEU). Coded \code{1} if it does and \code{0} if it does not.
\item[\code{existing\_aid}] \code{dummy}\hspace{5pt}A dummy variable indicating whether the case relates to existing state aid measures. Coded \code{1} if it does and \code{0} if it does not.
\item[\code{notified\_aid}] \code{dummy}\hspace{5pt}A dummy variable indicating whether the case relates to state aid notified by the member state. Coded \code{1} if it does and \code{0} if it does not.
\item[\code{unnotified\_aid}] \code{dummy}\hspace{5pt}A dummy variable indicating whether the case relates to state aid that is not notified by the member state. Coded \code{1} if it does and \code{0} if it does not.
\item[\code{general\_block\_exemption}] \code{dummy}\hspace{5pt}A dummy variable indicating whether the case relates to state that is exempt from notification under the General Block Exemption Regulation (GBER). Coded \code{1} if it does and \code{0} if it does not.
\item[\code{specific\_block\_exemption}] \code{dummy}\hspace{5pt}A dummy variable indicating whether the case relates to state aid that is exempt from notification under a specific block exemption. Coded \code{1} if it does and \code{0} if it does not.
\item[\code{notification\_date}] \code{date}\hspace{5pt}The date the member state notified the Commission of the state aid measure in the format \code{YYYY-MM-DD}.  
\item[\code{notification\_year}] \code{numeric}\hspace{5pt}The year the member state notified the Commission of the state aid measure.
\item[\code{notification\_month}] \code{numeric}\hspace{5pt}The month the member state notified the Commission of the state aid measure.
\item[\code{notification\_day}] \code{numeric}\hspace{5pt}The day the member state notified the Commission of the state aid measure.
\item[\code{outcome\_date}] \code{date}\hspace{5pt}The date the Commission took its last decision in the case in the format \code{YYYY-MM-DD}.
\item[\code{outcome\_year}] \code{numeric}\hspace{5pt}The year the Commission took its last decision in the case.
\item[\code{outcome\_month}] \code{numeric}\hspace{5pt}The month the Commission took its last decision in the case.
\item[\code{outcome\_day}] \code{numeric}\hspace{5pt}The day the Commission took its last decision in the case.
\item[\code{decisions}] \code{string}\hspace{5pt}A list of all of the decisions made by the Commission in the case, separated by a semicolon.
\item[\code{count\_decisions}] \code{numeric}\hspace{5pt}The total number of decisions made by the Commission in the case. 
\item[\code{outcome}] \code{string}\hspace{5pt}The outcome of the case. Possible values include: \code{Exempt from notification}, \code{Does not constitute aid}, \code{No objection}\code{Positive Decision}, \code{Negative Decision}, \code{Conditional decision}. Coded \code{Notification withdrawn} if the member state withdrew the notification, either before the formal investigation procedure under Article 10(1) of Council Regulation (EU) 2015/1589, formerly Article 8(1) of Council Regulation (EC) No 659/1999, or during the formal investigation procedure under Article 10(2) of Council Regulation (EU) 2015/1589, formerly Article 8(2) of Council Regulation (EC) No 659/1999. 
\item[\code{outcome\_phase\_1}] \code{string}\hspace{5pt}If the Commission opened a preliminary investigation, the outcome of that phase of the procedure. Possible values include: \code{Does not constitute aid}, \code{No objection}, \code{Formal investigation}, and \code{Notification withdrawn}. Coded \code{Not applicable} if the Commission did not open a preliminary investigation, which is when it determines that the state aid measure is exempt from notification under Article 108(2) of the Treaty on the Functioning of the European Union (TFEU).
\item[\code{outcome\_phase\_2}] \code{string}\hspace{5pt}If the Commission opened a formal investigation, the outcome of that phase of the procedure. Possible values include: \code{Conditional decision}, \code{Does not constitute aid}, \code{Negative decision}, \code{Notification withdrawn}, and \code{Positive decision}. Coded \code{Not applicable} if the Commission did not open a formal investigation under Article 4(4) of Council Regulation (EU) 2015/1589, formerly Article 4(4) of Council Regulation (EC) No 659/1999. Coded \code{Missing record} for a small number of observations where the Commission does not report data.
\item[\code{exempt}] \code{dummy}\hspace{5pt}A dummy variable indicating whether the state aid measure is exempt from notification under Article 108(2) of the Treaty on the Functioning of the European Union (TFEU).
\item[\code{preliminary\_investigation}] \code{dummy}\hspace{5pt}A dummy variable indicating whether the Commission decided to open a preliminary investigation under Article 108 of the Treaty on the Functioning of the European Union (TFEU).
\item[\code{formal\_investigation}] \code{dummy}\hspace{5pt}A dummy variable indicating whether the Commission decided to open a formal investigation under Article 4(4) of Council Regulation (EU) 2015/1589, formerly Article 4(4) of Council Regulation (EC) No 659/1999.
\item[\code{no\_objection}] \code{dummy}\hspace{5pt}A dummy variable indicating whether the Commission decided not to raise an objection to the state aid measure under Article 4(3) of Council Regulation (EU) 2015/1589, formerly Article 4(3) of Council Regulation (EC) No 659/1999. 
\item[\code{not\_aid}] \code{dummy}\hspace{5pt}A dummy variable indicating whether the Commission decided that the state aid measure did not constitute state aid, according to Article 107 of the Treaty on the Functioning of the European Union (TFEU), under Article 4(2) of Council Regulation (EU) 2015/1589, formerly Article 4(2) of Council Regulation (EC) No 659/1999. 
\item[\code{positive}] \code{dummy}\hspace{5pt}A dummy variable indicating whether, after a formal investigation, the Commission decided under article 9(3) of Council Regulation (EU) 2015/1589, formerly Article 7(3) of Council Regulation (EC) No 659/1999, to approve the state aid measure because it was compatible with the rules of the single market, as defined by Article 107 of the Treaty of the Functioning of the European Union (TFEU).
\item[\code{negative}] \code{dummy}\hspace{5pt}A dummy variable indicating whether, after a formal investigation, the Commission decided under article 9(5) of Council Regulation (EU) 2015/1589, formerly Article 7(5) of Council Regulation (EC) No 659/1999, to not approve the state aid measure because it was not compatible with the rules of the single market, as defined by Article 107 of the Treaty of the Functioning of the European Union (TFEU).
\item[\code{conditional}] \code{dummy}\hspace{5pt}A dummy variable indicating whether, after a formal investigation, the Commission decided under article 9(4) of Council Regulation (EU) 2015/1589, formerly Article 7(4) of Council Regulation (EC) No 659/1999, to approve the state aid measure subject to conditions to make the measure compatible with the rules of the single market, as defined by Article 107 of the Treaty of the Functioning of the European Union (TFEU).
\item[\code{withdrawal}] \code{dummy}\hspace{5pt}A dummy variable indicating whether the member state that notified the state aid measure withdrew the measure, either before the formal investigation procedure under Article 10(1) of Council Regulation (EU) 2015/1589, formerly Article 8(1) of Council Regulation (EC) No 659/1999, or during the formal investigation procedure under Article 10(2) of Council Regulation (EU) 2015/1589, formerly Article 8(2) of Council Regulation (EC) No 659/1999. 
\item[\code{referral}] \code{dummy}\hspace{5pt}A dummy variable indicating whether, after a formal investigation, the Commission decided to refer the state aid case to the Court of Justice of the European Union (CJEU) under Article 14 of Council Regulation (EU) 2015/1589, formerly Article 12 of Council Regulation (EC) No 659/1999, for noncompliance with an injunction, under Article 108(2) of the Treaty on the Functioning of the European Union (TFEU), for noncompliance with a Commission decision, or under Article 260(2) of the Treaty on the Functioning of the European Union (TFEU), for noncompliance with a judgment of the Court.
\item[\code{recovery}] \code{dummy}\hspace{5pt}A dummy variable indicating whether, after a formal investigation, the Commission decided under 16(1) of Council Regulation (EU) 2015/1589, formerly 14(1) of Council Regulation (EC) No 659/1999, that the member state that notified that state aid measure must take all necessary measures to recover aid from the beneficiary.
\end{description}
%--------------------------------------------------%
% dataset
%--------------------------------------------------%

\headerpage{cases\_ts}{Case-level time-series data}{30}{10}

\subheading{Description}

This dataset includes aggregated data on the number of state aid cases per year (time-series data). There is one observation per year (1988-2020).

\subheading{Variables}

\begin{description}[labelwidth=130pt, leftmargin=\dimexpr\labelwidth+\labelsep\relax, font=\normalfont, itemsep=10pt]
\item[\code{key\_id}] \code{numeric}\hspace{5pt}An ID number that uniquely identifies each observation in the dataset. 
\item[\code{year}] \code{numeric}\hspace{5pt}The year the case was opened by the Commission.
\item[\code{count\_cases}] \code{numeric}\hspace{5pt}A count of the number of cases opened by the Commission at this level of aggregation.
\end{description}
%--------------------------------------------------%
% dataset
%--------------------------------------------------%

\headerpage{cases\_ts\_ct}{Case-level time-series data by case type}{30}{10}

\subheading{Description}

This dataset includes aggregated data on the number of state aid cases per year (time-series data) broken down by case type. There is one observation per year per case type (1988-2020).

\subheading{Variables}

\begin{description}[labelwidth=130pt, leftmargin=\dimexpr\labelwidth+\labelsep\relax, font=\normalfont, itemsep=10pt]
\item[\code{key\_id}] \code{numeric}\hspace{5pt}An ID number that uniquely identifies each observation in the dataset. 
\item[\code{year}] \code{numeric}\hspace{5pt}The year the case was opened by the Commission.
\item[\code{case\_type\_id}] \code{numeric}\hspace{5pt}An ID number that uniquely identifies each type of state aid cases. Coded \code{1} for cases that involve state aid measures granted through Commission-approved schemes that is not notifiable, coded \code{2} for cases that involve state aid measures granted through Commission-approved schemes where the Commission has required that the member state notify any aid granted through the scheme, and coded \code{3} for cases that involve ad hoc state aid measures that are not granted through a scheme.
\item[\code{case\_type}] \code{string}\hspace{5pt}The type of the state aid case. There are three types of cases. Coded \code{Scheme} for cases that involve state aid measures granted through Commission-approved schemes that is not notifiable, coded \code{Individual application} for cases that involve state aid measures granted through Commission-approved schemes where the Commission has required that the member state notify any aid granted through the scheme, and coded \code{Ad hoc} for cases that involve ad hoc state aid measures that are not granted through a scheme.
\item[\code{count\_cases}] \code{numeric}\hspace{5pt}A count of the number of cases opened by the Commission at this level of aggregation.
\end{description}
%--------------------------------------------------%
% dataset
%--------------------------------------------------%

\headerpage{cases\_csts\_ms}{Case-level cross-sectional time-series data by member state}{30}{10}

\subheading{Description}

This dataset includes aggregated data on the number of state aid cases per member state per year (cross-sectional time-series data). There is one observation per member state per year (1988-2020), excluding state-years where the state was not a member of the EU.

\subheading{Variables}

\begin{description}[labelwidth=130pt, leftmargin=\dimexpr\labelwidth+\labelsep\relax, font=\normalfont, itemsep=10pt]
\item[\code{key\_id}] \code{numeric}\hspace{5pt}An ID number that uniquely identifies each observation in the dataset. 
\item[\code{year}] \code{numeric}\hspace{5pt}The year the case was opened by the Commission.
\item[\code{member\_state\_id}] \code{numeric}\hspace{5pt}An ID number that uniquely identifies each member state. This ID number is assigned when member states are sorted by accession date and then alphabetically. 
\item[\code{member\_state}] \code{string}\hspace{5pt}The name of the member state that the Commission opened the case against. 
\item[\code{member\_state\_code}] \code{string}\hspace{5pt}A two letter code assigned by the Commission that uniquely identifies each member state. 
\item[\code{count\_cases}] \code{numeric}\hspace{5pt}A count of the number of cases opened by the Commission at this level of aggregation.
\end{description}
%--------------------------------------------------%
% dataset
%--------------------------------------------------%

\headerpage{cases\_csts\_ms\_ct}{Case-level cross-sectional time-series data by member state and case type}{30}{10}

\subheading{Description}

This dataset includes aggregated data on the number of state aid cases per member state per year (cross-sectional time-series data) broken down by case type. There is one observation per member state per year per case type (1988-2020), excluding state-years where the state was not a member of the EU.

\subheading{Variables}

\begin{description}[labelwidth=130pt, leftmargin=\dimexpr\labelwidth+\labelsep\relax, font=\normalfont, itemsep=10pt]
\item[\code{key\_id}] \code{numeric}\hspace{5pt}An ID number that uniquely identifies each observation in the dataset. 
\item[\code{year}] \code{numeric}\hspace{5pt}The year the case was opened by the Commission.
\item[\code{member\_state\_id}] \code{numeric}\hspace{5pt}An ID number that uniquely identifies each member state. This ID number is assigned when member states are sorted by accession date and then alphabetically. 
\item[\code{member\_state}] \code{string}\hspace{5pt}The name of the member state that the Commission opened the case against. 
\item[\code{member\_state\_code}] \code{string}\hspace{5pt}A two letter code assigned by the Commission that uniquely identifies each member state. 
\item[\code{case\_type\_id}] \code{numeric}\hspace{5pt}An ID number that uniquely identifies each type of state aid cases. Coded \code{1} for cases that involve state aid measures granted through Commission-approved schemes that is not notifiable, coded \code{2} for cases that involve state aid measures granted through Commission-approved schemes where the Commission has required that the member state notify any aid granted through the scheme, and coded \code{3} for cases that involve ad hoc state aid measures that are not granted through a scheme.
\item[\code{case\_type}] \code{string}\hspace{5pt}The type of the state aid case. There are three types of cases. Coded \code{Scheme} for cases that involve state aid measures granted through Commission-approved schemes that is not notifiable, coded \code{Individual application} for cases that involve state aid measures granted through Commission-approved schemes where the Commission has required that the member state notify any aid granted through the scheme, and coded \code{Ad hoc} for cases that involve ad hoc state aid measures that are not granted through a scheme.
\item[\code{count\_cases}] \code{numeric}\hspace{5pt}A count of the number of cases opened by the Commission at this level of aggregation.
\end{description}
%--------------------------------------------------%
% dataset
%--------------------------------------------------%

\headerpage{cases\_csts\_dp}{Case-level cross-sectional time-series data by department}{30}{10}

\subheading{Description}

This dataset includes aggregated data on the number of state aid cases per department per year (cross-sectional time-series data). There is one observation per department per year (1988-2020). The dataset uses current department names. 

\subheading{Variables}

\begin{description}[labelwidth=130pt, leftmargin=\dimexpr\labelwidth+\labelsep\relax, font=\normalfont, itemsep=10pt]
\item[\code{key\_id}] \code{numeric}\hspace{5pt}An ID number that uniquely identifies each observation in the dataset. 
\item[\code{year}] \code{numeric}\hspace{5pt}The year the case was opened by the Commission.
\item[\code{department\_id}] \code{numeric}\hspace{5pt}An ID number that uniquely identifies each Directorate-General (DG) of the Commission. There are three DGs that can open state aid cases. The DG for Competition (COMP) is coded \code{1}, the DG for Agriculture and Rural Development (AGRI) is coded \code{2}, and the DG for Maritime Affairs and Fisheries (MARE) is coded \code{3}.
\item[\code{department}] \code{string}\hspace{5pt}The name of the Directorate-General (DG) of the Commission that opened the state aid case.
\item[\code{department\_code}] \code{string}\hspace{5pt}A multi-letter code assigned by the Commission that uniquely identifies each Directorate-General (DG) of the Commission. 
\item[\code{count\_cases}] \code{numeric}\hspace{5pt}A count of the number of cases opened by the Commission at this level of aggregation.
\end{description}
%--------------------------------------------------%
% dataset
%--------------------------------------------------%

\headerpage{cases\_csts\_dp\_ct}{Case-level cross-sectional time-series data by department and case type}{30}{10}

\subheading{Description}

This dataset includes aggregated data on the number of state aid cases per department per year (cross-sectional time-series data) broken down by case type. There is one observation per department per year per case type (1988-2020). The dataset uses current department names.

\subheading{Variables}

\begin{description}[labelwidth=130pt, leftmargin=\dimexpr\labelwidth+\labelsep\relax, font=\normalfont, itemsep=10pt]
\item[\code{key\_id}] \code{numeric}\hspace{5pt}An ID number that uniquely identifies each observation in the dataset. 
\item[\code{year}] \code{numeric}\hspace{5pt}The year the case was opened by the Commission.
\item[\code{department\_id}] \code{numeric}\hspace{5pt}An ID number that uniquely identifies each Directorate-General (DG) of the Commission. There are three DGs that can open state aid cases. The DG for Competition (COMP) is coded \code{1}, the DG for Agriculture and Rural Development (AGRI) is coded \code{2}, and the DG for Maritime Affairs and Fisheries (MARE) is coded \code{3}.
\item[\code{department}] \code{string}\hspace{5pt}The name of the Directorate-General (DG) of the Commission that opened the state aid case.
\item[\code{department\_code}] \code{string}\hspace{5pt}A multi-letter code assigned by the Commission that uniquely identifies each Directorate-General (DG) of the Commission. 
\item[\code{case\_type\_id}] \code{numeric}\hspace{5pt}An ID number that uniquely identifies each type of state aid cases. Coded \code{1} for cases that involve state aid measures granted through Commission-approved schemes that is not notifiable, coded \code{2} for cases that involve state aid measures granted through Commission-approved schemes where the Commission has required that the member state notify any aid granted through the scheme, and coded \code{3} for cases that involve ad hoc state aid measures that are not granted through a scheme.
\item[\code{case\_type}] \code{string}\hspace{5pt}The type of the state aid case. There are three types of cases. Coded \code{Scheme} for cases that involve state aid measures granted through Commission-approved schemes that is not notifiable, coded \code{Individual application} for cases that involve state aid measures granted through Commission-approved schemes where the Commission has required that the member state notify any aid granted through the scheme, and coded \code{Ad hoc} for cases that involve ad hoc state aid measures that are not granted through a scheme.
\item[\code{count\_cases}] \code{numeric}\hspace{5pt}A count of the number of cases opened by the Commission at this level of aggregation.
\end{description}
%--------------------------------------------------%
% dataset
%--------------------------------------------------%

\headerpage{cases\_ddy}{Case-level directed dyad-year data}{30}{10}

\subheading{Description}

This dataset includes aggregated data on the number of state aid cases per department per member state per year (directed dyad-year data). There is one observation per department per member state per year (1988-2020), excluding directed dyad-years where the state was not a member of the EU. The dataset uses current department names. 

\subheading{Variables}

\begin{description}[labelwidth=130pt, leftmargin=\dimexpr\labelwidth+\labelsep\relax, font=\normalfont, itemsep=10pt]
\item[\code{key\_id}] \code{numeric}\hspace{5pt}An ID number that uniquely identifies each observation in the dataset. 
\item[\code{year}] \code{numeric}\hspace{5pt}The year the case was opened by the Commission.
\item[\code{department\_id}] \code{numeric}\hspace{5pt}An ID number that uniquely identifies each Directorate-General (DG) of the Commission. There are three DGs that can open state aid cases. The DG for Competition (COMP) is coded \code{1}, the DG for Agriculture and Rural Development (AGRI) is coded \code{2}, and the DG for Maritime Affairs and Fisheries (MARE) is coded \code{3}.
\item[\code{department}] \code{string}\hspace{5pt}The name of the Directorate-General (DG) of the Commission that opened the state aid case.
\item[\code{department\_code}] \code{string}\hspace{5pt}A multi-letter code assigned by the Commission that uniquely identifies each Directorate-General (DG) of the Commission. 
\item[\code{member\_state\_id}] \code{numeric}\hspace{5pt}An ID number that uniquely identifies each member state. This ID number is assigned when member states are sorted by accession date and then alphabetically. 
\item[\code{member\_state}] \code{string}\hspace{5pt}The name of the member state that the Commission opened the case against. 
\item[\code{member\_state\_code}] \code{string}\hspace{5pt}A two letter code assigned by the Commission that uniquely identifies each member state. 
\item[\code{count\_cases}] \code{numeric}\hspace{5pt}A count of the number of cases opened by the Commission at this level of aggregation.
\end{description}
%--------------------------------------------------%
% dataset
%--------------------------------------------------%

\headerpage{cases\_ddy\_ct}{Case-level directed dyad-year data by case type}{30}{10}

\subheading{Description}

This dataset includes aggregated data on the number of state aid cases per department per member state per year (directed dyad-year data) broken down by case type. There is one observation per department per member state per year per case type (1988-2020), excluding directed dyad-years where the state was not a member of the EU. The dataset uses current department names. 

\subheading{Variables}

\begin{description}[labelwidth=130pt, leftmargin=\dimexpr\labelwidth+\labelsep\relax, font=\normalfont, itemsep=10pt]
\item[\code{key\_id}] \code{numeric}\hspace{5pt}An ID number that uniquely identifies each observation in the dataset. 
\item[\code{year}] \code{numeric}\hspace{5pt}The year the case was opened by the Commission.
\item[\code{department\_id}] \code{numeric}\hspace{5pt}An ID number that uniquely identifies each Directorate-General (DG) of the Commission. There are three DGs that can open state aid cases. The DG for Competition (COMP) is coded \code{1}, the DG for Agriculture and Rural Development (AGRI) is coded \code{2}, and the DG for Maritime Affairs and Fisheries (MARE) is coded \code{3}.
\item[\code{department}] \code{string}\hspace{5pt}The name of the Directorate-General (DG) of the Commission that opened the state aid case.
\item[\code{department\_code}] \code{string}\hspace{5pt}A multi-letter code assigned by the Commission that uniquely identifies each Directorate-General (DG) of the Commission. 
\item[\code{member\_state\_id}] \code{numeric}\hspace{5pt}An ID number that uniquely identifies each member state. This ID number is assigned when member states are sorted by accession date and then alphabetically. 
\item[\code{member\_state}] \code{string}\hspace{5pt}The name of the member state that the Commission opened the case against. 
\item[\code{member\_state\_code}] \code{string}\hspace{5pt}A two letter code assigned by the Commission that uniquely identifies each member state. 
\item[\code{case\_type\_id}] \code{numeric}\hspace{5pt}An ID number that uniquely identifies each type of state aid cases. Coded \code{1} for cases that involve state aid measures granted through Commission-approved schemes that is not notifiable, coded \code{2} for cases that involve state aid measures granted through Commission-approved schemes where the Commission has required that the member state notify any aid granted through the scheme, and coded \code{3} for cases that involve ad hoc state aid measures that are not granted through a scheme.
\item[\code{case\_type}] \code{string}\hspace{5pt}The type of the state aid case. There are three types of cases. Coded \code{Scheme} for cases that involve state aid measures granted through Commission-approved schemes that is not notifiable, coded \code{Individual application} for cases that involve state aid measures granted through Commission-approved schemes where the Commission has required that the member state notify any aid granted through the scheme, and coded \code{Ad hoc} for cases that involve ad hoc state aid measures that are not granted through a scheme.
\item[\code{count\_cases}] \code{numeric}\hspace{5pt}A count of the number of cases opened by the Commission at this level of aggregation.
\end{description}
%--------------------------------------------------%
% dataset
%--------------------------------------------------%

\headerpage{cases\_net}{Case-level network data}{30}{10}

\subheading{Description}

This dataset includes network data for state aid cases. Network data is similar to directed dyad-year data except that it only includes directed dyad-years with at least one infringement case. For every year, there is one node per department and one node per member state. Edges can only exist between a department and a member state. There is an edge between a department and a member state if and only if the department opened at least one case against the member state during that year. The weight of the edge is the number of cases that the department opened against the member state. There is one observation per department per member state per year (1988-2020), excluding directed dyad-years where the state was not a member of the EU, but only if count of cases is positive.

\subheading{Variables}

\begin{description}[labelwidth=130pt, leftmargin=\dimexpr\labelwidth+\labelsep\relax, font=\normalfont, itemsep=10pt]
\item[\code{key\_id}] \code{numeric}\hspace{5pt}An ID number that uniquely identifies each observation in the dataset. 
\item[\code{year}] \code{numeric}\hspace{5pt}The year the case was opened by the Commission.
\item[\code{from\_node\_id}] \code{numeric}\hspace{5pt}An ID number that uniquely identifies each node in the network that creates a link, which is always a Commission department.
\item[\code{from\_node}] \code{string}\hspace{5pt}The name of the Commission department that opened the case.
\item[\code{to\_node\_id}] \code{numeric}\hspace{5pt}An ID number that uniquely identifies each node in the network that receives a link, which is always a member state.
\item[\code{to\_node}] \code{string}\hspace{5pt}The name of the member state that the Commission opened the case against. 
\item[\code{edge\_weight}] \code{numeric}\hspace{5pt}The weight of the edge, which is the number of cases opened by the Commission.
\end{description}
%--------------------------------------------------%
% dataset
%--------------------------------------------------%

\headerpage{cases\_net\_ct}{Case-level network data by case type}{30}{10}

\subheading{Description}

This dataset includes multi-dimensional network data for state aid cases. There is one dimension per case type. Network data is similar to directed dyad-year data except that it only includes directed dyad-years with at least one infringement case with respect to each case type. For every year, there is one node per department and one node per member state. Edges can only exist between a department and a member state. There is an edge between a department and a member state with respect to each case type if and only if the department opened at least one case against the member state during that year. The weight of the edge is the number of cases that the department opened against the member state. There is one observation per department per member state per year per case type (1988-2020), excluding directed dyad-years where the state was not a member of the EU, but only if the count of cases is positive.

\subheading{Variables}

\begin{description}[labelwidth=130pt, leftmargin=\dimexpr\labelwidth+\labelsep\relax, font=\normalfont, itemsep=10pt]
\item[\code{key\_id}] \code{numeric}\hspace{5pt}An ID number that uniquely identifies each observation in the dataset. 
\item[\code{year}] \code{numeric}\hspace{5pt}The year the case was opened by the Commission.
\item[\code{layer\_id}] \code{numeric}\hspace{5pt}An ID number that uniquely identifies each layer of the network.
\item[\code{layer}] \code{string}\hspace{5pt}The layer of the network, which is type of case.
\item[\code{from\_node\_id}] \code{numeric}\hspace{5pt}An ID number that uniquely identifies each node in the network that creates a link, which is always a Commission department.
\item[\code{from\_node}] \code{string}\hspace{5pt}The name of the Commission department that opened the case.
\item[\code{to\_node\_id}] \code{numeric}\hspace{5pt}An ID number that uniquely identifies each node in the network that receives a link, which is always a member state.
\item[\code{to\_node}] \code{string}\hspace{5pt}The name of the member state that the Commission opened the case against. 
\item[\code{edge\_weight}] \code{numeric}\hspace{5pt}The weight of the edge, which is the number of cases opened by the Commission.
\end{description}
%--------------------------------------------------%
% dataset
%--------------------------------------------------%

\headerpage{decisions}{Decision-level data}{30}{10}

\subheading{Description}

This dataset include data on decisions in state aid cases. There is one observation per decision per case (1988-2020). The dataset includes information on the department responsible for the decision and the member state that the decision is against.

\subheading{Variables}

\begin{description}[labelwidth=130pt, leftmargin=\dimexpr\labelwidth+\labelsep\relax, font=\normalfont, itemsep=10pt]
\item[\code{key\_id}] \code{numeric}\hspace{5pt}An ID number that uniquely identifies each observation in the dataset. 
\item[\code{case\_id}] \code{string}\hspace{5pt}An ID number that uniquely identifies each state aid case. Assigned by the Commission. The Commission changed the format of case numbers in 2010. Before this change, the case number indicated the type(s) of the procedure associated with the case. After this change, all case numbers have the format \code{SA.\#\#\#\#\#}.
\item[\code{procedure\_numbers}] \code{string}\hspace{5pt}Numbers that uniquely identifiy each procedure associated with a case. If there are multiple procedures, the procedure numbers are listed and separated by a comma. After the Commission changed the format of case numbers in 2010, only procedure numbers indicate the types of procedures associated with a case.
\item[\code{member\_state\_id}] \code{numeric}\hspace{5pt}An ID number that uniquely identifies each member state. This ID number is assigned when member states are sorted by accession date and then alphabetically. 
\item[\code{member\_state}] \code{string}\hspace{5pt}The name of the member state that the Commission opened the case against. 
\item[\code{member\_state\_code}] \code{string}\hspace{5pt}A two letter code assigned by the Commission that uniquely identifies each member state. 
\item[\code{department\_id}] \code{numeric}\hspace{5pt}An ID number that uniquely identifies each Directorate-General (DG) of the Commission. There are three DGs that can open state aid cases. The DG for Competition (COMP) is coded \code{1}, the DG for Agriculture and Rural Development (AGRI) is coded \code{2}, and the DG for Maritime Affairs and Fisheries (MARE) is coded \code{3}.
\item[\code{department}] \code{string}\hspace{5pt}The name of the Directorate-General (DG) of the Commission that opened the state aid case.
\item[\code{department\_code}] \code{string}\hspace{5pt}A multi-letter code assigned by the Commission that uniquely identifies each Directorate-General (DG) of the Commission. 
\item[\code{case\_type\_id}] \code{numeric}\hspace{5pt}An ID number that uniquely identifies each type of state aid cases. Coded \code{1} for cases that involve state aid measures granted through Commission-approved schemes that is not notifiable, coded \code{2} for cases that involve state aid measures granted through Commission-approved schemes where the Commission has required that the member state notify any aid granted through the scheme, and coded \code{3} for cases that involve ad hoc state aid measures that are not granted through a scheme.
\item[\code{case\_type}] \code{string}\hspace{5pt}The type of the state aid case. There are three types of cases. Coded \code{Scheme} for cases that involve state aid measures granted through Commission-approved schemes that is not notifiable, coded \code{Individual application} for cases that involve state aid measures granted through Commission-approved schemes where the Commission has required that the member state notify any aid granted through the scheme, and coded \code{Ad hoc} for cases that involve ad hoc state aid measures that are not granted through a scheme.
\item[\code{procedure\_types}] \code{string}\hspace{5pt}Numbers that uniquely identified each procedure associated with a case. If there are multiple procedures, the procedure numbers are listed and separated by a comma. After the Commission changed the format of case numbers in 2010, only procedure numbers indicate the types of procedures associated with a case.
\item[\code{contradictory\_aid}] \code{dummy}\hspace{5pt}A dummy variable indicating whether the case relates to whether the Commission suspects that the state aid measure is contradictory to the rules of the single market, as defined by Article 107 of the Treaty of the Functioning of the European Union (TFEU). Coded \code{1} if it does and \code{0} if it does not.
\item[\code{existing\_aid}] \code{dummy}\hspace{5pt}A dummy variable indicating whether the case relates to existing state aid measures. Coded \code{1} if it does and \code{0} if it does not.
\item[\code{notified\_aid}] \code{dummy}\hspace{5pt}A dummy variable indicating whether the case relates to state aid notified by the member state. Coded \code{1} if it does and \code{0} if it does not.
\item[\code{unnotified\_aid}] \code{dummy}\hspace{5pt}A dummy variable indicating whether the case relates to state aid that is not notified by the member state. Coded \code{1} if it does and \code{0} if it does not.
\item[\code{general\_block\_exemption}] \code{dummy}\hspace{5pt}A dummy variable indicating whether the case relates to state that is exempt from notification under the General Block Exemption Regulation (GBER). Coded \code{1} if it does and \code{0} if it does not.
\item[\code{specific\_block\_exemption}] \code{dummy}\hspace{5pt}A dummy variable indicating whether the case relates to state aid that is exempt from notification under a specific block exemption. Coded \code{1} if it does and \code{0} if it does not.
\item[\code{notification\_date}] \code{date}\hspace{5pt}The date the member state notified the Commission of the state aid measure in the format \code{YYYY-MM-DD}.  
\item[\code{notification\_year}] \code{numeric}\hspace{5pt}The year the member state notified the Commission of the state aid measure.
\item[\code{notification\_month}] \code{numeric}\hspace{5pt}The month the member state notified the Commission of the state aid measure.
\item[\code{notification\_day}] \code{numeric}\hspace{5pt}The day the member state notified the Commission of the state aid measure.
\item[\code{outcome\_date}] \code{date}\hspace{5pt}The date the Commission took its last decision in the case in the format \code{YYYY-MM-DD}.
\item[\code{outcome\_year}] \code{numeric}\hspace{5pt}The year the Commission took its last decision in the case.
\item[\code{outcome\_month}] \code{numeric}\hspace{5pt}The month the Commission took its last decision in the case.
\item[\code{outcome\_day}] \code{numeric}\hspace{5pt}The day the Commission took its last decision in the case.
\item[\code{decision\_number}] \code{numeric}\hspace{5pt}The number of the decision within each case.
\item[\code{decision\_type\_id}] \code{numeric}\hspace{5pt}An ID number that uniquely identifies each type of decision the Commission can make in a state aid case.
\item[\code{decision\_type}] \code{string}\hspace{5pt}The type of the decision.
\item[\code{phase}] \code{string}\hspace{5pt}The phase of the procedure. Coded \code{Exempt} for state aid that is exempt from notification, \code{Phase 1} for decisions made during a preliminary investigation, \code{Phase 2} for decisions made during a formal investigation, or \code{Other} for decisions that the Commission can make during either phase. 
\end{description}
%--------------------------------------------------%
% dataset
%--------------------------------------------------%

\headerpage{decisions\_ts}{Decision-level time-series data}{30}{10}

\subheading{Description}

This dataset includes aggregated data on the number of each type of decision per year (time-series data). There is one observation per year per decision stage (1988-2020).

\subheading{Variables}

\begin{description}[labelwidth=130pt, leftmargin=\dimexpr\labelwidth+\labelsep\relax, font=\normalfont, itemsep=10pt]
\item[\code{key\_id}] \code{numeric}\hspace{5pt}An ID number that uniquely identifies each observation in the dataset. 
\item[\code{year}] \code{numeric}\hspace{5pt}The year the case was opened by the Commission.
\item[\code{decision\_type\_id}] \code{numeric}\hspace{5pt}An ID number that uniquely identifies each type of decision the Commission can make in a state aid case.
\item[\code{decision\_type}] \code{string}\hspace{5pt}The type of the decision.
\item[\code{count\_decisions}] \code{numeric}\hspace{5pt}A count of the number of decisions issued by the Commission at this level of aggregation.
\end{description}
%--------------------------------------------------%
% dataset
%--------------------------------------------------%

\headerpage{decisions\_ts\_ct}{Decision-level time-series data by case type}{30}{10}

\subheading{Description}

This dataset includes aggregated data on the number of each type of decision per year (time-series data) broken down by case type. There is one observation per year per decision type per case type (1988-2020).

\subheading{Variables}

\begin{description}[labelwidth=130pt, leftmargin=\dimexpr\labelwidth+\labelsep\relax, font=\normalfont, itemsep=10pt]
\item[\code{key\_id}] \code{numeric}\hspace{5pt}An ID number that uniquely identifies each observation in the dataset. 
\item[\code{year}] \code{numeric}\hspace{5pt}The year the case was opened by the Commission.
\item[\code{case\_type\_id}] \code{numeric}\hspace{5pt}An ID number that uniquely identifies each type of state aid cases. Coded \code{1} for cases that involve state aid measures granted through Commission-approved schemes that is not notifiable, coded \code{2} for cases that involve state aid measures granted through Commission-approved schemes where the Commission has required that the member state notify any aid granted through the scheme, and coded \code{3} for cases that involve ad hoc state aid measures that are not granted through a scheme.
\item[\code{case\_type}] \code{string}\hspace{5pt}The type of the state aid case. There are three types of cases. Coded \code{Scheme} for cases that involve state aid measures granted through Commission-approved schemes that is not notifiable, coded \code{Individual application} for cases that involve state aid measures granted through Commission-approved schemes where the Commission has required that the member state notify any aid granted through the scheme, and coded \code{Ad hoc} for cases that involve ad hoc state aid measures that are not granted through a scheme.
\item[\code{decision\_type\_id}] \code{numeric}\hspace{5pt}An ID number that uniquely identifies each type of decision the Commission can make in a state aid case.
\item[\code{decision\_type}] \code{string}\hspace{5pt}The type of the decision.
\item[\code{count\_cases}] \code{numeric}\hspace{5pt}A count of the number of cases opened by the Commission at this level of aggregation.
\end{description}
%--------------------------------------------------%
% dataset
%--------------------------------------------------%

\headerpage{decisions\_csts\_ms}{Decision-level cross-sectional time-series data by member state}{30}{10}

\subheading{Description}

This dataset includes aggregated data on the number of each type of decision per member state per year (cross-sectional time-series data). There is one observation per member state per year per decision type (1988-2020), excluding state-years where the state was not a member of the EU.

\subheading{Variables}

\begin{description}[labelwidth=130pt, leftmargin=\dimexpr\labelwidth+\labelsep\relax, font=\normalfont, itemsep=10pt]
\item[\code{key\_id}] \code{numeric}\hspace{5pt}An ID number that uniquely identifies each observation in the dataset. 
\item[\code{year}] \code{numeric}\hspace{5pt}The year the case was opened by the Commission.
\item[\code{member\_state\_id}] \code{numeric}\hspace{5pt}An ID number that uniquely identifies each member state. This ID number is assigned when member states are sorted by accession date and then alphabetically. 
\item[\code{member\_state}] \code{string}\hspace{5pt}The name of the member state that the Commission opened the case against. 
\item[\code{member\_state\_code}] \code{string}\hspace{5pt}A two letter code assigned by the Commission that uniquely identifies each member state. 
\item[\code{decision\_type\_id}] \code{numeric}\hspace{5pt}An ID number that uniquely identifies each type of decision the Commission can make in a state aid case.
\item[\code{decision\_type}] \code{string}\hspace{5pt}The type of the decision.
\item[\code{count\_decisions}] \code{numeric}\hspace{5pt}A count of the number of decisions issued by the Commission at this level of aggregation.
\end{description}
%--------------------------------------------------%
% dataset
%--------------------------------------------------%

\headerpage{decisions\_csts\_ms\_ct}{Decision-level cross-sectional time-series data by member state and case type}{30}{10}

\subheading{Description}

This dataset includes aggregated data on the number of each type of decision per member state per year (cross-sectional time-series data) broken down by case type (noncommunication vs nonconformity). There is one observation per member state per year per decision type per case type (1988-2020), excluding state-years where the state was not a member of the EU.

\subheading{Variables}

\begin{description}[labelwidth=130pt, leftmargin=\dimexpr\labelwidth+\labelsep\relax, font=\normalfont, itemsep=10pt]
\item[\code{key\_id}] \code{numeric}\hspace{5pt}An ID number that uniquely identifies each observation in the dataset. 
\item[\code{year}] \code{numeric}\hspace{5pt}The year the case was opened by the Commission.
\item[\code{member\_state\_id}] \code{numeric}\hspace{5pt}An ID number that uniquely identifies each member state. This ID number is assigned when member states are sorted by accession date and then alphabetically. 
\item[\code{member\_state}] \code{string}\hspace{5pt}The name of the member state that the Commission opened the case against. 
\item[\code{member\_state\_code}] \code{string}\hspace{5pt}A two letter code assigned by the Commission that uniquely identifies each member state. 
\item[\code{case\_type\_id}] \code{numeric}\hspace{5pt}An ID number that uniquely identifies each type of state aid cases. Coded \code{1} for cases that involve state aid measures granted through Commission-approved schemes that is not notifiable, coded \code{2} for cases that involve state aid measures granted through Commission-approved schemes where the Commission has required that the member state notify any aid granted through the scheme, and coded \code{3} for cases that involve ad hoc state aid measures that are not granted through a scheme.
\item[\code{case\_type}] \code{string}\hspace{5pt}The type of the state aid case. There are three types of cases. Coded \code{Scheme} for cases that involve state aid measures granted through Commission-approved schemes that is not notifiable, coded \code{Individual application} for cases that involve state aid measures granted through Commission-approved schemes where the Commission has required that the member state notify any aid granted through the scheme, and coded \code{Ad hoc} for cases that involve ad hoc state aid measures that are not granted through a scheme.
\item[\code{decision\_type\_id}] \code{numeric}\hspace{5pt}An ID number that uniquely identifies each type of decision the Commission can make in a state aid case.
\item[\code{decision\_type}] \code{string}\hspace{5pt}The type of the decision.
\item[\code{count\_decisions}] \code{numeric}\hspace{5pt}A count of the number of decisions issued by the Commission at this level of aggregation.
\end{description}
%--------------------------------------------------%
% dataset
%--------------------------------------------------%

\headerpage{decisions\_csts\_dp}{Decision-level cross-sectional time-series data by department}{30}{10}

\subheading{Description}

This dataset includes aggregated data on the number of each type of decision per department per year (cross-sectional time-series data). There is one observation per department per year per decision type (1988-2020). The dataset uses current department names. 

\subheading{Variables}

\begin{description}[labelwidth=130pt, leftmargin=\dimexpr\labelwidth+\labelsep\relax, font=\normalfont, itemsep=10pt]
\item[\code{key\_id}] \code{numeric}\hspace{5pt}An ID number that uniquely identifies each observation in the dataset. 
\item[\code{year}] \code{numeric}\hspace{5pt}The year the case was opened by the Commission.
\item[\code{department\_id}] \code{numeric}\hspace{5pt}An ID number that uniquely identifies each Directorate-General (DG) of the Commission. There are three DGs that can open state aid cases. The DG for Competition (COMP) is coded \code{1}, the DG for Agriculture and Rural Development (AGRI) is coded \code{2}, and the DG for Maritime Affairs and Fisheries (MARE) is coded \code{3}.
\item[\code{department}] \code{string}\hspace{5pt}The name of the Directorate-General (DG) of the Commission that opened the state aid case.
\item[\code{department\_code}] \code{string}\hspace{5pt}A multi-letter code assigned by the Commission that uniquely identifies each Directorate-General (DG) of the Commission. 
\item[\code{decision\_type\_id}] \code{numeric}\hspace{5pt}An ID number that uniquely identifies each type of decision the Commission can make in a state aid case.
\item[\code{decision\_type}] \code{string}\hspace{5pt}The type of the decision.
\item[\code{count\_decisions}] \code{numeric}\hspace{5pt}A count of the number of decisions issued by the Commission at this level of aggregation.
\end{description}
%--------------------------------------------------%
% dataset
%--------------------------------------------------%

\headerpage{decisions\_csts\_dp\_ct}{Decision-level cross-sectional time-series data by department and case type}{30}{10}

\subheading{Description}

This dataset includes aggregated data on the number of each type of decision per department per year (cross-sectional time-series data) broken down by case type (noncommunication vs nonconformity). There is one observation per department per year per decision type per case type (1988-2020). The dataset uses current department names. 

\subheading{Variables}

\begin{description}[labelwidth=130pt, leftmargin=\dimexpr\labelwidth+\labelsep\relax, font=\normalfont, itemsep=10pt]
\item[\code{key\_id}] \code{numeric}\hspace{5pt}An ID number that uniquely identifies each observation in the dataset. 
\item[\code{year}] \code{numeric}\hspace{5pt}The year the case was opened by the Commission.
\item[\code{department\_id}] \code{numeric}\hspace{5pt}An ID number that uniquely identifies each Directorate-General (DG) of the Commission. There are three DGs that can open state aid cases. The DG for Competition (COMP) is coded \code{1}, the DG for Agriculture and Rural Development (AGRI) is coded \code{2}, and the DG for Maritime Affairs and Fisheries (MARE) is coded \code{3}.
\item[\code{department}] \code{string}\hspace{5pt}The name of the Directorate-General (DG) of the Commission that opened the state aid case.
\item[\code{department\_code}] \code{string}\hspace{5pt}A multi-letter code assigned by the Commission that uniquely identifies each Directorate-General (DG) of the Commission. 
\item[\code{case\_type\_id}] \code{numeric}\hspace{5pt}An ID number that uniquely identifies each type of state aid cases. Coded \code{1} for cases that involve state aid measures granted through Commission-approved schemes that is not notifiable, coded \code{2} for cases that involve state aid measures granted through Commission-approved schemes where the Commission has required that the member state notify any aid granted through the scheme, and coded \code{3} for cases that involve ad hoc state aid measures that are not granted through a scheme.
\item[\code{case\_type}] \code{string}\hspace{5pt}The type of the state aid case. There are three types of cases. Coded \code{Scheme} for cases that involve state aid measures granted through Commission-approved schemes that is not notifiable, coded \code{Individual application} for cases that involve state aid measures granted through Commission-approved schemes where the Commission has required that the member state notify any aid granted through the scheme, and coded \code{Ad hoc} for cases that involve ad hoc state aid measures that are not granted through a scheme.
\item[\code{decision\_type\_id}] \code{numeric}\hspace{5pt}An ID number that uniquely identifies each type of decision the Commission can make in a state aid case.
\item[\code{decision\_type}] \code{string}\hspace{5pt}The type of the decision.
\item[\code{count\_decisions}] \code{numeric}\hspace{5pt}A count of the number of decisions issued by the Commission at this level of aggregation.
\end{description}
%--------------------------------------------------%
% dataset
%--------------------------------------------------%

\headerpage{decisions\_ddy}{Decision-level directed dyad-year data}{30}{10}

\subheading{Description}

This dataset includes aggregated data on the number of each type of decision per department per member state per year (directed dyad-year data). There is one observation per department per member state per year per decision type (1988-2020), excluding directed dyad-years where the state was not a member of the EU. The dataset uses current department names. 

\subheading{Variables}

\begin{description}[labelwidth=130pt, leftmargin=\dimexpr\labelwidth+\labelsep\relax, font=\normalfont, itemsep=10pt]
\item[\code{key\_id}] \code{numeric}\hspace{5pt}An ID number that uniquely identifies each observation in the dataset. 
\item[\code{year}] \code{numeric}\hspace{5pt}The year the case was opened by the Commission.
\item[\code{department\_id}] \code{numeric}\hspace{5pt}An ID number that uniquely identifies each Directorate-General (DG) of the Commission. There are three DGs that can open state aid cases. The DG for Competition (COMP) is coded \code{1}, the DG for Agriculture and Rural Development (AGRI) is coded \code{2}, and the DG for Maritime Affairs and Fisheries (MARE) is coded \code{3}.
\item[\code{department}] \code{string}\hspace{5pt}The name of the Directorate-General (DG) of the Commission that opened the state aid case.
\item[\code{department\_code}] \code{string}\hspace{5pt}A multi-letter code assigned by the Commission that uniquely identifies each Directorate-General (DG) of the Commission. 
\item[\code{member\_state\_id}] \code{numeric}\hspace{5pt}An ID number that uniquely identifies each member state. This ID number is assigned when member states are sorted by accession date and then alphabetically. 
\item[\code{member\_state}] \code{string}\hspace{5pt}The name of the member state that the Commission opened the case against. 
\item[\code{member\_state\_code}] \code{string}\hspace{5pt}A two letter code assigned by the Commission that uniquely identifies each member state. 
\item[\code{decision\_type\_id}] \code{numeric}\hspace{5pt}An ID number that uniquely identifies each type of decision the Commission can make in a state aid case.
\item[\code{decision\_type}] \code{string}\hspace{5pt}The type of the decision.
\item[\code{count\_decisions}] \code{numeric}\hspace{5pt}A count of the number of decisions issued by the Commission at this level of aggregation.
\end{description}
%--------------------------------------------------%
% dataset
%--------------------------------------------------%

\headerpage{decisions\_ddy\_ct}{Decision-level directed dyad-year data by case type}{30}{10}

\subheading{Description}

This dataset includes aggregated data on the number of each type of decision per department per member state per year (directed dyad-year data) broken down by case type (noncommunication vs nonconformity). There is one observation per department per member state per year per case type per decision type (1988-2020), excluding directed dyad-years where the state was not a member of the EU. The dataset uses current department names. 

\subheading{Variables}

\begin{description}[labelwidth=130pt, leftmargin=\dimexpr\labelwidth+\labelsep\relax, font=\normalfont, itemsep=10pt]
\item[\code{key\_id}] \code{numeric}\hspace{5pt}An ID number that uniquely identifies each observation in the dataset. 
\item[\code{year}] \code{numeric}\hspace{5pt}The year the case was opened by the Commission.
\item[\code{department\_id}] \code{numeric}\hspace{5pt}An ID number that uniquely identifies each Directorate-General (DG) of the Commission. There are three DGs that can open state aid cases. The DG for Competition (COMP) is coded \code{1}, the DG for Agriculture and Rural Development (AGRI) is coded \code{2}, and the DG for Maritime Affairs and Fisheries (MARE) is coded \code{3}.
\item[\code{department}] \code{string}\hspace{5pt}The name of the Directorate-General (DG) of the Commission that opened the state aid case.
\item[\code{department\_code}] \code{string}\hspace{5pt}A multi-letter code assigned by the Commission that uniquely identifies each Directorate-General (DG) of the Commission. 
\item[\code{member\_state\_id}] \code{numeric}\hspace{5pt}An ID number that uniquely identifies each member state. This ID number is assigned when member states are sorted by accession date and then alphabetically. 
\item[\code{member\_state}] \code{string}\hspace{5pt}The name of the member state that the Commission opened the case against. 
\item[\code{member\_state\_code}] \code{string}\hspace{5pt}A two letter code assigned by the Commission that uniquely identifies each member state. 
\item[\code{case\_type\_id}] \code{numeric}\hspace{5pt}An ID number that uniquely identifies each type of state aid cases. Coded \code{1} for cases that involve state aid measures granted through Commission-approved schemes that is not notifiable, coded \code{2} for cases that involve state aid measures granted through Commission-approved schemes where the Commission has required that the member state notify any aid granted through the scheme, and coded \code{3} for cases that involve ad hoc state aid measures that are not granted through a scheme.
\item[\code{case\_type}] \code{string}\hspace{5pt}The type of the state aid case. There are three types of cases. Coded \code{Scheme} for cases that involve state aid measures granted through Commission-approved schemes that is not notifiable, coded \code{Individual application} for cases that involve state aid measures granted through Commission-approved schemes where the Commission has required that the member state notify any aid granted through the scheme, and coded \code{Ad hoc} for cases that involve ad hoc state aid measures that are not granted through a scheme.
\item[\code{decision\_type\_id}] \code{numeric}\hspace{5pt}An ID number that uniquely identifies each type of decision the Commission can make in a state aid case.
\item[\code{decision\_type}] \code{string}\hspace{5pt}The type of the decision.
\item[\code{count\_decisions}] \code{numeric}\hspace{5pt}A count of the number of decisions issued by the Commission at this level of aggregation.
\end{description}
%--------------------------------------------------%
% dataset
%--------------------------------------------------%

\headerpage{decisions\_net}{Decision-level network data}{30}{10}

\subheading{Description}

This dataset includes multi-dimensional network data for decisions in state aid cases. There is one dimension per decision type. Network data is similar to directed dyad-year data except that it only includes directed dyad-years with at least one decision. For every year, there is one node per department and one node per member state. Edges can only exist between a department and a member state. There is an edge between a department and a member state with respect to each decision type if and only if the department issued at least one decision against the member state during that year. The weight of the edge is the number of decisions that the department opened against the member state. There is one observation per department per member state per year per decision type (1988-2020), excluding directed dyad-years where the state was not a member of the EU, but only if count of decisions is positive.

\subheading{Variables}

\begin{description}[labelwidth=130pt, leftmargin=\dimexpr\labelwidth+\labelsep\relax, font=\normalfont, itemsep=10pt]
\item[\code{key\_id}] \code{numeric}\hspace{5pt}An ID number that uniquely identifies each observation in the dataset. 
\item[\code{year}] \code{numeric}\hspace{5pt}The year the decision was issued by the Commission.
\item[\code{layer\_id}] \code{numeric}\hspace{5pt}An ID number that uniquely identifies each layer of the network.
\item[\code{layer}] \code{string}\hspace{5pt}The layer of the network, which is the decision type.
\item[\code{from\_node\_id}] \code{numeric}\hspace{5pt}An ID number that uniquely identifies each node in the network that creates a link, which is always a Commission department.
\item[\code{from\_node}] \code{string}\hspace{5pt}The name of the Commission department that opened the case.
\item[\code{to\_node\_id}] \code{numeric}\hspace{5pt}An ID number that uniquely identifies each node in the network that receives a link, which is always a member state.
\item[\code{to\_node}] \code{string}\hspace{5pt}The name of the member state that the Commission issued the decision against. 
\item[\code{edge\_weight}] \code{numeric}\hspace{5pt}The weight of the edge, which is the number of decisions opened by the Commission.
\end{description}
%--------------------------------------------------%
% dataset
%--------------------------------------------------%

\headerpage{decisions\_net\_ct}{Decision-level network data by case type}{30}{10}

\subheading{Description}

This dataset includes multi-dimensional network data for infringement decisions. There is one dimension per case type and per decision type. Network data is similar to directed dyad-year data except that it only includes directed dyad-years with at least one decision. For every year, there is one node per department and one node per member state. Edges can only exist between a department and a member state. There is an edge between a department and a member state with respect to each case type and decision type if and only if the department issued at least one decision against the member state during that year. The weight of the edge is the number of decisions that the department opened against the member state. There is one observation per department per member state per year per case type per decision type (1988-2020), excluding directed dyad-years where the state was not a member of the EU, but only if the count of decisions is positive.

\subheading{Variables}

\begin{description}[labelwidth=130pt, leftmargin=\dimexpr\labelwidth+\labelsep\relax, font=\normalfont, itemsep=10pt]
\item[\code{key\_id}] \code{numeric}\hspace{5pt}An ID number that uniquely identifies each observation in the dataset. 
\item[\code{year}] \code{numeric}\hspace{5pt}The year the decision was issued by the Commission.
\item[\code{d1\_layer\_id}] \code{numeric}\hspace{5pt}An ID number that uniquely identifies each layer of the network within the first dimension.
\item[\code{d1\_layer}] \code{string}\hspace{5pt}The layer of the network within the first dimension, which is type of case.
\item[\code{d2\_layer\_id}] \code{numeric}\hspace{5pt}An ID number that uniquely identifies each layer of the network within the second dimension.
\item[\code{d2\_layer}] \code{string}\hspace{5pt}The layer of the network within the second dimension, which is the decision type.
\item[\code{from\_node\_id}] \code{numeric}\hspace{5pt}An ID number that uniquely identifies each node in the network that creates a link, which is always a Commission department.
\item[\code{from\_node}] \code{string}\hspace{5pt}The name of the Commission department that opened the case.
\item[\code{to\_node\_id}] \code{numeric}\hspace{5pt}An ID number that uniquely identifies each node in the network that receives a link, which is always a member state.
\item[\code{to\_node}] \code{string}\hspace{5pt}The name of the member state that the Commission opened the case against. 
\item[\code{edge\_weight}] \code{numeric}\hspace{5pt}The weight of the edge, which is the number of decisions opened by the Commission.
\end{description}
%--------------------------------------------------%
% dataset
%--------------------------------------------------%

\headerpage{awards}{Award-level data}{30}{10}

\subheading{Description}

This dataset includes data on state aid awards granted to firms by government agencies in member states that were reported to the Commission. There is one observation per award (2016-2020). The dataset includes information on the date of the award, the aid-granting agency, the beneficiary, the NACE sector of the beneficiary, the aid instrument used, and the estimated amount of the aid in euros. 

\subheading{Variables}

\begin{description}[labelwidth=130pt, leftmargin=\dimexpr\labelwidth+\labelsep\relax, font=\normalfont, itemsep=10pt]
\item[\code{key\_id}] \code{numeric}\hspace{5pt}An ID number that uniquely identifies each observation in the dataset. 
\item[\code{case\_id}] \code{string}\hspace{5pt}An ID number that uniquely identifies each state aid case. Assigned by the Commission. The Commission changed the format of case numbers in 2010. Before this change, the case number indicated the type(s) of the procedure associated with the case. After this change, all case numbers have the format \code{SA.\#\#\#\#\#}.
\item[\code{reference\_number}] \code{string}\hspace{5pt}An ID number that uniquely identifies each state aid award.
\item[\code{notification\_date}] \code{date}\hspace{5pt}The date the member state notified the Commission of the state aid measure in the format \code{YYYY-MM-DD}.  
\item[\code{notification\_year}] \code{numeric}\hspace{5pt}The year the member state notified the Commission of the state aid measure.
\item[\code{notification\_month}] \code{numeric}\hspace{5pt}The month the member state notified the Commission of the state aid measure.
\item[\code{notification\_day}] \code{numeric}\hspace{5pt}The day the member state notified the Commission of the state aid measure.
\item[\code{publication\_date}] \code{date}\hspace{5pt}The date that the Commission published a record of the state aid award in the format \code{YYYY-MM-DD}.
\item[\code{publication\_year}] \code{numeric}\hspace{5pt}The year that the Commission published a record of the state aid award.
\item[\code{publication\_month}] \code{numeric}\hspace{5pt}The month that the Commission published a record of the state aid award.
\item[\code{publication\_day}] \code{numeric}\hspace{5pt}The day that the Commission published a record of the state aid award.
\item[\code{member\_state\_id}] \code{numeric}\hspace{5pt}An ID number that uniquely identifies each member state. This ID number is assigned when member states are sorted by accession date and then alphabetically. 
\item[\code{member\_state}] \code{string}\hspace{5pt}The name of the member state that the Commission opened the case against. 
\item[\code{member\_state\_code}] \code{string}\hspace{5pt}A two letter code assigned by the Commission that uniquely identifies each member state. 
\item[\code{authority\_name}] \code{string}\hspace{5pt}The name of the national, subnational, or local authority that granted the state aid award. Note that the text of this variable has not been cleaned.
\item[\code{region}] \code{string}\hspace{5pt}The region of the member state where the beneficiary of the state aid award is located. Note that the text of this variable has not been cleaned.
\item[\code{beneficiary\_name}] \code{string}\hspace{5pt}The name of the beneficiary of the state aid award. Note that the text of this variable has not been cleaned.
\item[\code{beneficiary\_type\_id}] \code{numeric}\hspace{5pt}An ID number that uniquely identifies each type of beneficiary. Coded \code{1} for \code{Small or medium-sized enterprise (SME)} and \code{2} for \code{Large enterprise}.
\item[\code{beneficiary\_type}] \code{string}\hspace{5pt}The type of the beneficiary of the state aid award. Coded \code{Small or medium-sized enterprise (SME)} or \code{Large enterprise}. Uses the Commission’s standard definition of small and median-sized enterprises (SMEs). 
\item[\code{nace\_sector\_id}] \code{numeric}\hspace{5pt}An ID number that uniquely identifies each NACE sector.
\item[\code{nace\_sector}] \code{string}\hspace{5pt}The NACE sector for the state aid award.
\item[\code{nace\_sector\_code}] \code{string}\hspace{5pt}The NACE sector code for the state aid award (a single capital letter).
\item[\code{nace\_code}] \code{string}\hspace{5pt}The NACE code for the state aid award.
\item[\code{nace\_description}] \code{string}\hspace{5pt}A description of the NACE code for the state aid award.
\item[\code{aid\_instrument\_id}] \code{numeric}\hspace{5pt}An ID number that uniquely identifies each state aid instrument that a granting authority can use.
\item[\code{aid\_instrument}] \code{string}\hspace{5pt}The aid instrument that the granting authority used in granting the state aid award.
\item[\code{raw\_amount}] \code{numeric}\hspace{5pt}The amount of the state aid award expressed in units of the local currency. If the value of the award is expressed as a range, this variable records the mean of the range. 
\item[\code{currency}] \code{string}\hspace{5pt}The name of the currency in which the member state reported the amount of the state aid award.
\item[\code{range}] \code{dummy}\hspace{5pt}A dummy variable indicating whether the amount of the state aid award is expressed as a range instead of a specific value (allowed by the Commission under some circumstances to protect proprietary business data). Coded \code{1} if the amount is expressed as a range and \code{0} otherwise.
\item[\code{range\_min}] \code{numeric}\hspace{5pt}If the value of the state aid award is expressed as a range (see \code{range}), the minimum value of the range expressed in units of the local currency. Coded \code{NA} if the value of the award is not expressed as a range.
\item[\code{range\_max}] \code{numeric}\hspace{5pt}If the value of the state aid award is expressed as a range (see \code{range}), the maximum value of the range expressed in units of the local currency. Coded \code{NA} if the value of the award is not expressed as a range.
\item[\code{exchange\_rate}] \code{numeric}\hspace{5pt}The exchange rate for converting the local currency into euros. Coded \code{1} if the local currency is the euro.
\item[\code{amount\_euros}] \code{numeric}\hspace{5pt}The total estimated value of the award in euros.
\item[\code{voluntary}] \code{dummy}\hspace{5pt}A dummy variable indicating whether the state aid award was small enough that notification to the Commission is voluntary. Coded \code{1} if notification is voluntary and \code{0} otherwise.
\end{description}
%--------------------------------------------------%
% dataset
%--------------------------------------------------%

\headerpage{awards\_csts}{Award-level cross-sectional time-series data}{30}{10}

\subheading{Description}

This dataset includes aggregated data on the number of state aid awards per member state per year (cross-sectional time-series data). There is one observation per member state per year (2016-2020).

\subheading{Variables}

\begin{description}[labelwidth=130pt, leftmargin=\dimexpr\labelwidth+\labelsep\relax, font=\normalfont, itemsep=10pt]
\item[\code{key\_id}] \code{numeric}\hspace{5pt}An ID number that uniquely identifies each observation in the dataset. 
\item[\code{year}] \code{numeric}\hspace{5pt}The year the case was opened by the Commission.
\item[\code{member\_state\_id}] \code{numeric}\hspace{5pt}An ID number that uniquely identifies each member state. This ID number is assigned when member states are sorted by accession date and then alphabetically. 
\item[\code{member\_state}] \code{string}\hspace{5pt}The name of the member state that the Commission opened the case against. 
\item[\code{member\_state\_code}] \code{string}\hspace{5pt}A two letter code assigned by the Commission that uniquely identifies each member state. 
\item[\code{count\_awards}] \code{numeric}\hspace{5pt}A count of the number of state aid awards granted by the member state at this level of aggregation.
\item[\code{total\_amount\_awards}] \code{numeric}\hspace{5pt}The total amount of awards granted at this level of aggregation in euros.
\end{description}
%--------------------------------------------------%
% dataset
%--------------------------------------------------%

\headerpage{awards\_csts\_bt}{Award-level cross-sectional time-series data by beneficiary type}{30}{10}

\subheading{Description}

This dataset includes aggregated data on the number of state aid awards per member state per year (cross-sectional time-series data) broken down by beneficiary type. There is one observation per member state per year per beneficiary type (2016-2020).

\subheading{Variables}

\begin{description}[labelwidth=130pt, leftmargin=\dimexpr\labelwidth+\labelsep\relax, font=\normalfont, itemsep=10pt]
\item[\code{key\_id}] \code{numeric}\hspace{5pt}An ID number that uniquely identifies each observation in the dataset. 
\item[\code{year}] \code{numeric}\hspace{5pt}The year the case was opened by the Commission.
\item[\code{member\_state\_id}] \code{numeric}\hspace{5pt}An ID number that uniquely identifies each member state. This ID number is assigned when member states are sorted by accession date and then alphabetically. 
\item[\code{member\_state}] \code{string}\hspace{5pt}The name of the member state that the Commission opened the case against. 
\item[\code{member\_state\_code}] \code{string}\hspace{5pt}A two letter code assigned by the Commission that uniquely identifies each member state. 
\item[\code{beneficiary\_type\_id}] \code{numeric}\hspace{5pt}An ID number that uniquely identifies each type of beneficiary. Coded \code{1} for \code{Small or medium-sized enterprise (SME)} and \code{2} for \code{Large enterprise}.
\item[\code{beneficiary\_type}] \code{string}\hspace{5pt}The type of the beneficiary of the state aid award. Coded \code{Small or medium-sized enterprise (SME)} or \code{Large enterprise}. Uses the Commission’s standard definition of small and median-sized enterprises (SMEs). 
\item[\code{count\_awards}] \code{numeric}\hspace{5pt}A count of the number of state aid awards granted by the member state at this level of aggregation.
\item[\code{total\_amount\_awards}] \code{numeric}\hspace{5pt}The total amount of awards granted at this level of aggregation in euros.
\end{description}
%--------------------------------------------------%
% dataset
%--------------------------------------------------%

\headerpage{awards\_csts\_ai}{Award-level cross-sectional time-series data by aid instrument}{30}{10}

\subheading{Description}

This dataset includes aggregated data on the number of state aid awards per member state per year (cross-sectional time-series data) broken down by aid instrument. There is one observation per member state per year per aid instrument (2016-2020).

\subheading{Variables}

\begin{description}[labelwidth=130pt, leftmargin=\dimexpr\labelwidth+\labelsep\relax, font=\normalfont, itemsep=10pt]
\item[\code{key\_id}] \code{numeric}\hspace{5pt}An ID number that uniquely identifies each observation in the dataset. 
\item[\code{year}] \code{numeric}\hspace{5pt}The year the case was opened by the Commission.
\item[\code{member\_state\_id}] \code{numeric}\hspace{5pt}An ID number that uniquely identifies each member state. This ID number is assigned when member states are sorted by accession date and then alphabetically. 
\item[\code{member\_state}] \code{string}\hspace{5pt}The name of the member state that the Commission opened the case against. 
\item[\code{member\_state\_code}] \code{string}\hspace{5pt}A two letter code assigned by the Commission that uniquely identifies each member state. 
\item[\code{aid\_instrument\_id}] \code{numeric}\hspace{5pt}An ID number that uniquely identifies each state aid instrument that a granting authority can use.
\item[\code{aid\_instrument}] \code{string}\hspace{5pt}The aid instrument that the granting authority used in granting the state aid award.
\item[\code{count\_awards}] \code{numeric}\hspace{5pt}A count of the number of state aid awards granted by the member state at this level of aggregation.
\item[\code{total\_amount\_awards}] \code{numeric}\hspace{5pt}The total amount of awards granted at this level of aggregation in euros.
\end{description}
%--------------------------------------------------%
% dataset
%--------------------------------------------------%

\headerpage{awards\_csts\_ns}{Award-level cross-sectional time-series data by NACE sector}{30}{10}

\subheading{Description}

This dataset includes aggregated data on the number of state aid awards per member state per year (cross-sectional time-series data) broken down by NACE sector. There is one observation per member state per year per NACE sector (2016-2020).

\subheading{Variables}

\begin{description}[labelwidth=130pt, leftmargin=\dimexpr\labelwidth+\labelsep\relax, font=\normalfont, itemsep=10pt]
\item[\code{key\_id}] \code{numeric}\hspace{5pt}An ID number that uniquely identifies each observation in the dataset. 
\item[\code{year}] \code{numeric}\hspace{5pt}The year the case was opened by the Commission.
\item[\code{member\_state\_id}] \code{numeric}\hspace{5pt}An ID number that uniquely identifies each member state. This ID number is assigned when member states are sorted by accession date and then alphabetically. 
\item[\code{member\_state}] \code{string}\hspace{5pt}The name of the member state that the Commission opened the case against. 
\item[\code{member\_state\_code}] \code{string}\hspace{5pt}A two letter code assigned by the Commission that uniquely identifies each member state. 
\item[\code{nace\_sector\_id}] \code{numeric}\hspace{5pt}An ID number that uniquely identifies each NACE sector.
\item[\code{nace\_sector}] \code{string}\hspace{5pt}The NACE sector for the state aid award.
\item[\code{nace\_sector\_code}] \code{string}\hspace{5pt}The NACE sector code for the state aid award (a single capital letter).
\item[\code{count\_awards}] \code{numeric}\hspace{5pt}A count of the number of state aid awards granted by the member state at this level of aggregation.
\item[\code{total\_amount\_awards}] \code{numeric}\hspace{5pt}The total amount of awards granted at this level of aggregation in euros.
\end{description}

%--------------------------------------------------%
% end document
%--------------------------------------------------%

\end{flushleft}

\end{document}
